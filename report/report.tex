\documentclass[12pt,a4paper]{scrartcl}
\usepackage[utf8]{inputenc}
\usepackage[english,russian]{babel}
\usepackage{indentfirst}
\usepackage{misccorr}
\usepackage{graphicx}
\usepackage{amsmath}
\usepackage[usenames]{color}
\usepackage{colortbl}
\usepackage{url}
\usepackage[colorlinks,unicode]{hyperref}
\begin{document}
%\maketitle


\begin{titlepage}
    \begin{center}
        \LARGE
        Практическое задание
  
        %«Исследование процесса установление равновесия для ансамбля сосудов на примере газа Лоренца».
        «Проверка эргодической гипотезы на примере газа Лоренца».
    \end{center}

    \par\bigskip
    
    \begin{center}
        \Large
        Курс «Введение в термодинамику и статистическую физику».
    \end{center}

	\begin{center}
		\Large
		Авторы: 
		
		Лунин Дмитрий, 317 группа
		
		Никишин Евгений, 317 группа
	\end{center}
	
	\begin{center}
		\Large
		Руководитель: 
		
		доцент Полякова Марина Сергеевна
	\end{center}

    \par\bigskip

    \tableofcontents
    
    \par\bigskip
    \par\bigskip
    \par\bigskip
    
    \par\bigskip
    \par\bigskip
    \par\bigskip
    \par\bigskip
    \par\bigskip
    \par\bigskip
    \par\bigskip
    \par\bigskip
    \par\bigskip
    \par\bigskip
    \par\bigskip
    \par\bigskip
    \par\bigskip
    
	\begin{center}
		\Large
		Москва, 2015
	\end{center}
\end{titlepage}


\section{Теоретическая часть. Эргодическая гипотеза.}
	
    В статистической физике для вероятностного описания термодинамической системы вводят понятие статистического ансамбля. Это совокупность очень большого числа идентичных термодинамических систем с одинаковыми внешними термодинамическими параметрами, но отличающихся друг от друга микросостоянием, т.е. значениями координат и импульсов частиц, составляющих систему. При таком подходе термодинамическая система представляется точкой в фазовом пространстве (пространстве координат и импульсов всех частиц системы $\vec{z}$), а статистический ансамбль --- "облаком"\ точек в фазовом пространстве. Функция распределения вероятностей $w(\vec{z})$ для фазовых переменных $\vec{z}$ является фактически нормированной на $1$ плотностью частиц с фазовом пространстве, а полученные с её помощью средние значения внутренних термодинамических параметров
    \begin{equation}
    \label{eq::vecB}
    \\<B \\>\ = \int{B(\vec{z})w(\vec{z})d\vec{z}}
    \end{equation}
называют фазовыми средними, или средними по ансамблю.

	При экспериментальных исследованиях, как правило, повторные измерения проводятся на одной системе и определяется временное среднее значение внутреннего термодинамического параметра
	\begin{equation}
    \label{eq::avB}
    \tilde{B} = \lim_{T \to \infty} \frac{1}{T} \int_{0}^{T} B(t)dt.
	\end{equation}
	
	Усреднение по времени может быть осуществлено лишь по конечному интервалу времени. Усреднение по формуле \eqref{eq::avB} предполагает, что интервал усреднения $T$ много больше всех характерных времен рассматриваемой системы, в частности, времени релаксационных процессов, приводящих к установлению равновесного состояния. Благодаря этому предельное значение $\tilde{B}$, определяемое формулой \eqref{eq::avB}, не зависит от распределения координат и импульсов частиц системы в начальный момент времени.
	
	Вопрос о соотношении временных и фазовых средних возник в самом начале развития статистической физики. При этом отправным пунктом служила так называемоя эргодическая гипотеза, согласно которой для равновесного состояния временные и фазовые средние внутренних параметров совпадают, т.е.
	\begin{equation}
	\label{eq::veceqav}
	\tilde{B} = \\<B \\>.
	\end{equation}
	
	%Внесение условий, при которых такое равенство имеет место, является одной из проблем современной эргодической теории. 
	В настоящей работе на примере газа Лоренца вам следует убедиться в справедливости эргодической гипотезы.
	
	Пусть в двумерном евклидовом пространстве, случайно разбросано бесконечное множество шаров (рассеивателей). Между ними случайно разбросано большое число точечных частиц. Величина скорости каждой частицы постоянна и одинакова для всех частиц. При столкновении с рассеивателем частица отражается от него по закону упругого удара. Масса рассеивателя много больше массы частицы. 
	
	Газ Лоренца из $n$ частиц --- это хорошая модель для отслеживания свойств термодиначеской системы. Так как наша модель не может содержать очень много частиц, то за термодинамическую систему будем принимать систему, содержащую 100\ - 150 частиц.
	
	В качестве внутреннего термодинамического параметра выбирается давление на стенку сосуда, вычисляемое как средний импульс силы, передаваемый за малое по сравнению с временем между последовательными соударениями частицы с рассеивателями время.
    
\section{Схема работы с программой}
        
        Для выполнения задания следует использовать программу-визуализатор, которая позволяет промоделировать движение частиц в модели газа Лоренца.

    В правой части экрана находятся окна, в которых можно задавать следующие параметры системы:

    \begin{itemize}
    	\item Количество электронов (от 0 до 9999)
        \item Скорость электронов (от 0 до 1000)
        \item Направление начальной скорости электронов, то есть угол между скоростью частицы и выбранной осью (от 0 до 360 градусов)
        \item Радиусы шаров (рассеивателей) (от 0 до 33)
        \item Количество систем в ансамбле (от 1 до 10000)
		\item На экран выводится поведение частиц среди рассеивателей одной из систем ансамбля. Можно посмотреть на любую из них (от 0 до n - 1, где n --- количество систем в ансамбле)
    \end{itemize}

	Для демонстрации необходимо задать, как минимум, количество электронов, остальные параметры будут иметь значения, набранные ранее в соответствующих окнах.
    
	Зависимость изменения давления со временем иллюстрируется графиками, которые рисуются в левом нижнем углу экрана. Синим обозначается среднее по ансамблю, зеленым --- среднее по времени, красным --- давление в системе, демонстрируемой на экране.
	
	Когда устанавливается равновесие, программа приостанавливается, и на экране появляется время установления равновесия.
	
	{\bf Критерий остановки}. Система вычисляет разность давления в демонстрируемой на экране системе и среднего давления по ансамблю и усредняет это значение по времени. Когда модуль этой разности становится меньше порога (экспериментально подобранного создателями программы и зависящего от количества электронов), программа приостанавливается.

\section{Упражнения}
\subsection{Наблюдение установления состояния равновесия в системе}
	Предлагается несколько раз запустить моделирование для того, чтобы удостовериться в том факте, что система приходит в равновесие.
	
	В программе показаны графики:
	\begin{enumerate}
	\item График давления в наблюдаемой на экране системе (красный).
	\item Усредненный по времени график давления в этой системе (зеленый).
	\item Усредненный по ансамблю график давления в системах (синий).
	\end{enumerate}
	
	По этим графикам можно оценить момент наступления равновесия в выбранной системе (см. критерий остановки).
\subsection{Исследование зависимости времени установления равновесия в системе от параметров}
	{\bf Примечание}. Удобно одновременно выполнять упражнения 2 и 3. Поэтому проделайте сначала 1-2 раза эти упражнения и только затем выполняйте их.
	
	Время наступления равновесия зависит от разных параметров, например, от количества электронов, радиусов атомов, начального расположения и направления начальных скоростей частиц. Чем более в начальный момент система далека от равновесного состояния, тем больше время его достижения. По умолчанию в программе начальные условия для частиц задаются случайным образом. Однако вы можете задавать начальное положение и начальную скорость каждой частицы с экрана. Примером сильно неравновесного состояния системы в начальный момент времени будет расположение частиц в малой по сравнению со всей доступной частицам области экрана. Предлагается исследовать, от каких параметров и как зависит время наступления равновесия. Выберите 2 набора начальных условий, при которых времена установления равновесия будут сильно отличаться (в 5-10 раз). Запишите эти времена и соответствующие начальные условия, так как для них вы будете выполнять упражнение 3.
	
\subsection{Сравнение средних давлений (по ансамблю и по времени)}
	Как уже было сказано, после установления равновесия в системе программа останавливается. Однако моделирование можно продолжить, снова нажав на клавишу ''Запуск''. Вы увидите, что средние давления по ансамблю и по времени сближаются. Можно посмотреть и качественно оценить, через какое время они начнут совпадать.
	
    
\section{Требования к оформлению отчёта}
    Решение практического задания должно быть оформлено в виде отчета, содержащего указание на авторство, конкретное задание, данное преподавателем, и результаты измерений (в виде краткого текста, графиков, таблиц).
	
% Показать, что вообще устанавливается равновесие
% Зависимость от параметров
% 1/sqrt(N)

       %%%\item Найти вид функции, выражающей зависимость начальных данных (угла, радиусов). 
        %%%\item Оценить изменение начального угла между двумя частицами, запущенными из одной точки, за время, через которой частицы удалятся друг от друга на заданное расстояние (момент останова демонстрации). 
        %%%\item Как влияет размер радиусов на изменение углов между частицами и временем работы демонстрации?
        %%%\item Записать геометрическую зависимость параметров.
\end{document}
